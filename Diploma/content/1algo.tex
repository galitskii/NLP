\documentclass[main]{subfiles}





\begin{document}

\subsection{Подготовительный этап}
На вход программе подаётся предложение, состоящее из существительных, местоимений, личных глаголов или (и) инфинитивов (с, возможно, перечислением инфинитивов или личных глаголов с зависимыми словами, принадлежащим указанным частям речи). 

Полученное предложение передаётся функции space(), которая преобразует считанную строку в список. Механизм её работы описывает алгоритм \ref{alg1}.

\begin{algorithm}
	\caption{-- Предварительная обработка входных данных}\label{alg1}
	\begin{algorithmic}[1]
		\Function{space}{str1}
		\State str1 $\gets$ str1.lower() \Comment{Приводим полученную строку к нижнему регистру}
		\State str2 $\gets $ <<>> \Comment{В этой переменной будет храниться преобразованная строка}
		\State l $\gets$ len(str1)
		\For{\textbf{from} $i=0$ \textbf{to} $l - 1$} 
		\If{str1$[i] \in \{$<<.>>; <<,>> $\}$  }
		\State str2 $\gets$ str2 $+$ << >>
		\EndIf
		\State str2 $\gets$ str2 $+$ str1$[i]$
		\EndFor
		\If{str2[len(str2) $- 1$] $=$ << >>} \Comment{Если последним элементом полученного списка оказался пробел}
		\State str2 $\gets$ str2[ : len(str2) $- 1$] \Comment{Отбрасываем этот пробел}
		\EndIf
		\State\Return str2.split() \Comment{Возвращаем список, полученный из строки str2 разбиением её по пробелам}
		\EndFunction
	\end{algorithmic}
\end{algorithm}

Таким образом, функция space() возвращает список, состоящий из слов и знаков препинания исходной строки.

\subsection{Обработка предложения}

Итак, как было сказано выше, проверка согласования единственного и множественного числа в русском языке~--- процесс сложный: нужно учесть много критериев. 

В основе предложенного нами подхода лежит гипотеза, согласно которой одно и то же предложение являться и не являться ошибочным одновременно с точки зрения согласования единственного и множественного числа не может. 

Также считаем, что в предложении нет орфографических, пунктуационных и др. ошибок, поскольку данная задача была успешно решена, например, компанией LanguageTooler GmbH \cite{langt}.

Нами было принято решение декомпозировать задачу. 

Для начала (при наличии перечислений инфинитивов или личных личных глаголов) предложение упрощается: перечисление мы заменяем на инфинитив или личный глагол соответственно (параллельно проверяя, что внутри заменяемой части нет ошибок в согласовании единственного и множественного числа). Если же перечисления не обнаружено, сразу переходим к следующему этапу.

Затем проверяем предложение без перечислений при помощи разработанной нами системы правил. 

Согласно теореме Гёделя о неполноте, формальная арифметика либо противоречива, либо неполна \cite{forms}. Чтобы избежать противоречивости разработанной системы, мы включили лишь те правила, которые встречаются на практике, а не перебрали все возможные комбинации используемых нами параметров.

\subsubsection{Упрощение предложения}
Под упрощением мы будем понимать замену перечисления инфинитивов или личных глаголов одиночным инфинитивом или личным глаголом. 

За упрощение предложения отвечает функция comma(), которая принимает на вход список, полученный из исходного предложения при помощи функции space(), описанной выше; а возвращает список, в виде которого представлено упрощённое предложение. Механизм работы функции comma() описывает алгоритм \ref{alg2}.

\begin{algorithm}
	\caption{-- Обработка перечислений}\label{alg2}
	\begin{algorithmic}[1]
		\Function{comma}{l} \Comment{l~--- подготовленная строка в виде списка}
		\If{<<и>> \textbf{in} l}
		\State part $\gets [$ $]$ \Comment{Список значений параметра <<часть речи>> для данного слова (изначально пустой)}
		\State end$\gets (-1)$ \Comment{Индикатор нахождения начала перечисления}
		\State left $\gets (-1)$ \Comment{Левая и правая границы заменяемого участка изначально}
		\State right $\gets (-1)$\Comment{инициализируем невозможными значениями: $(-1)$}
		\State llen $\gets $len(l) \Comment{Длина исходного списка}
		\State id1 $\gets$ l.index(<<и>>)\Comment{Записываем индекс <<и>> в списке} 
		\algstore{bkbreak}
	\end{algorithmic}
\end{algorithm}
Для начала инициализируем переменные, затем находим индекс вхождения <<и>> в список (при условии, что в списке есть <<и>>). После этого анализируем слова, находящиеся в окрестности слова <<и>>:

\begin{algorithm}
	\caption{-- Продолжение алгоритма \ref{alg2}}\label{alg3}
	\begin{algorithmic}[1]
		\algrestore{bkbreak}
				\If{llen $>$ id1 $+ 1$}
		\State resp1 $\gets$ l[id1 $+ 1$].[pos, singular, cow] \Comment{Варианты интерпретации слова, стоящего за <<и>>}
		\EndIf
		\If{llen $>$ id1 $+ 2$}
		\State resp2 $\gets$ l[id1 $+ 2$].[pos, singular, cow]
		\EndIf	
		\If{llen $>$ id1 $+ 3$}
		\State resp3 $\gets$ l[id1 $+ 3$].[pos, singular, cow] 
		\EndIf
		\If{llen $>$ id1 $+ 4$}
		\State resp4 $\gets$ l[id1 $+ 4$].[pos, singular, cow]
		\EndIf
		\If{id1 $-1 \ge 0$}
		\State resl1 $\gets$ l[id1 $-1$].[pos, singular, cow]
		\EndIf
		\If{id1 $-2 \ge 0$}
		\State resl2 $\gets$ l[id1 $-2$].[pos, singular, cow]
		\EndIf
		\If{id1 $-3 \ge 0$}
		\State resl3 $\gets$ l[id1 $-3$].[pos, singular, cow]
		\EndIf
		\If{id1 $-4 \ge 0$}
		\State resl1 $\gets$ l[id1 $-4$].[pos, singular, cow]
		\EndIf
		\If{id1 $-5 \ge 0$}
		\State resl5 $\gets$ l[id1 $-5$].[pos, singular, cow]
		\EndIf
		\If{id1 $-6 \ge 0$}
		\State resl6 $\gets$ l[id1 $-6$].[pos, singular, cow]
		\EndIf
		\If{id1 $+1 <$ llen}
		\For{\textbf{from} $i=0$ \textbf{to} len(resp1)$- 1$}
		\If{resp1$[i][0]=$<<6>>} \Comment{Слово оказалось инфинитивом}
		\State part$\gets$ <<6>>
		\State right $\gets$ id1 $+1$
				\EndIf
					\EndFor
					\If{right$=(-1)$}\Comment{Если же это не инфинитив}
					\For{\textbf{from} $i=0$ \textbf{to} len(resp1)$-1$}
					\If{resp1$[i][0]=$<<5>>}\Comment{Слово оказалось личным глаголом}
					\State right $\gets$id1$ + 1$ 
					\State part $\gets$ <<5>>
					\State sng $\gets $l$[i][1]$ \Comment{Для личных глаголов важно число}
					\State \textbf{break}
							\EndIf
							\EndFor
									\EndIf
											\EndIf
		\algstore{bkbreak}
	\end{algorithmic}
\end{algorithm}

Таким образом, в результате исполнения блока \ref{alg3} будет определено, слова (словосочетания) какой части речи перечисляются (если в предложении присутствует перечисление с союзом <<и>>).

Заметим, что в случае перечисления с союзом <<и>> за союзом идёт слово той же части речи, что и остальные перечисляемые слова. Например: \textit{<<Он хотел \textbf{читать} книги, \textbf{рисовать} картины и \textbf{познавать} тайны мироздания>>}. Легко видеть, что в предложении перечисляются инфинитивы, и в то же время после союза <<и>> идёт инфинитив <<познавать>>.

Если перечисляются инфинитивы, то упрощение предложения идёт согласно алгоритму \ref{alg4}.

Прежде всего, нужно определить начало левого операнда <<и>>. В зависимости от длины буквосочетания, возможны различные варианты:
\begin{enumerate}
	\item Словосочетание длины $6$. Например, инф. + сущ. + сущ. + сущ. + сущ. + сущ.: \textit{<<Организовать проверку знаний требований охраны труда>>}.
	\item Словосочетание длины $5$. Например, инф. + сущ. + сущ. + сущ. + сущ.: \textit{<<Организовать проверку знаний основ программирования>>}.
	\item Словосочетание длины $4$. Например, инф. + инф. + сущ. + сущ.: \textit{<<Пойти спать сном младенца>>}.
	\item Словосочетание длины $3$. Например, инф. + сущ. + сущ.: \textit{<<Оценить игру слов>>}.
	\item Словосочетание длины $2$. Например, инф. + инф.: \textit{<<Пойти позавтракать>>}.
	\item Одиночный инфинитив. Например: \textit{<<Быть>>}.
\end{enumerate}
Итак, первым делом инициализируем левую границу заменяемого <<куска>> списка.
\begin{algorithm}
	\caption{-- Продолжение алгоритма \ref{alg3}}\label{alg4}
	\begin{algorithmic}[1]
		\algrestore{bkbreak}
		\If{part $=$<<6>>}\Comment{Если перечисляемая часть речи~--- инфинитив}
		\If{id1$-6\ge 0$ \textbf{and} left$=(-1)$ \textbf{and} <<,>>$\notin$ l$[$id1$ - 6:\,$ id1$]$} \Comment{Проверяем буквосочетания длины $6$}
		\For{\textbf{from} $i = 0$ \textbf{to} len(resl6)$-1$}
		\If{resl6$[i][0]=$part}
		\State r$\gets$ check(l$[$id1$-6$ : id1$]$)
		\If{<<N>>$\in $ r}
		\State \Return $[$<<он>>, <<писали>> $]$ \Comment{Заведомо неверное предложение}
		\algstore{bkbreak}
	\end{algorithmic}
\end{algorithm}

\begin{algorithm}
	\caption{-- Продолжение алгоритма \ref{alg4}}\label{alg5}
	\begin{algorithmic}[1]
		\algrestore{bkbreak}
		\ElsIf{<<Y>>$\in $ r}
		\State left $\gets$ id1$-6$ \Comment{Инициализировали границу левого операнда <<и>>}
		\State \textbf{break}
		\EndIf
		\EndIf
		\EndFor
		\EndIf
		\If{id1$-5\ge 0$ \textbf{and} left$=-1$ \textbf{and} <<,>>$\notin$ l$[$id1$ - 5:\,$ id1$]$}
		\For{\textbf{from} $i = 0$ \textbf{to} len(resl5)$-1$}
		\If{resl5$[i][0]=$part}
		\State r$\gets$ check(l$[$id1$-5$ : id1$]$)
		\If{<<N>>$\in $ r}
		\State \Return $[$<<он>>, <<писали>> $]$
		\ElsIf{<<Y>>$\in $ r}
		\State left $\gets$ id1$-5$ 
		\State \textbf{break}
		\EndIf
		\EndIf
		\EndFor
		\EndIf
		\If{id1$-4\ge 0$ \textbf{and} left$=-1$ \textbf{and} <<,>>$\notin$ l$[$id1$ - 4:\,$ id1$]$}
		\For{\textbf{from} $i = 0$ \textbf{to} len(resl4)$-1$}
		\If{resl4$[i][0]=$part}
		\State r$\gets$ check(l$[$id1$-4$ : id1$]$)
		\If{<<N>>$\in $ r}
		\State \Return $[$<<он>>, <<писали>> $]$
		\ElsIf{<<Y>>$\in $ r}
		\State left $\gets$ id1$-4$ 
		\State \textbf{break}
		\EndIf
		\EndIf
		\EndFor
		\EndIf
		\If{id1$-3\ge 0$ \textbf{and} left$=-1$ \textbf{and} <<,>>$\notin$ l$[$id1$ - 3:\,$ id1$]$}
		\For{\textbf{from} $i = 0$ \textbf{to} len(resl3)$-1$}
		\If{resl3$[i][0]=$part}
		\State r$\gets$ check(l$[$id1$-3$ : id1$]$)
	\If{<<N>>$\in $ r}
\State \Return $[$<<он>>, <<писали>> $]$
\ElsIf{<<Y>>$\in $ r}
\State left $\gets$ id1$-3$ 
\State \textbf{break}
\EndIf
\EndIf
\EndFor
		\EndIf
		\algstore{bkbreak}
	\end{algorithmic}
\end{algorithm}
Таким образом, в результате выполнения данного фрагмента кода будет определены границы левого и правого операндов <<и>>.
\begin{algorithm}
	\caption{-- Продолжение алгоритма \ref{alg5}}\label{alg6}
	\begin{algorithmic}[1]
		\algrestore{bkbreak}	
		\If{id1$-2\ge 0$ \textbf{and} left$=-1$ \textbf{and} <<,>>$\notin$ l$[$id1$ - 2:\,$ id1$]$}
		\For{\textbf{from} $i = 0$ \textbf{to} len(resl2)$-1$}
		\If{resl2$[i][0]=$part}
		\State r$\gets$ check(l$[$id1$-2$ : id1$]$)
		\If{<<N>>$\in $ r}
		\State \Return $[$<<он>>, <<писали>> $]$
		\ElsIf{<<Y>>$\in $ r}
		\State left $\gets$ id1$-2$ 
		\State \textbf{break}
		\EndIf
		\EndIf
		\EndFor
		\EndIf
		\If{id1$-1\ge 0$ \textbf{and} left$=(-1)$}
		\For{\textbf{from} $i = 0$ \textbf{to} len(resl1)$-1$}
		\If{l$[i][0]=$part}
		\State left $\gets$ id1$-1$
		\State \textbf{break}
		\EndIf
		\EndFor
		\EndIf
		\EndIf
		\State Алгоритм \ref{alg7}
		\State Алгоритм \ref{alg11}
		\EndIf
		\State Алгоритм \ref{alg12}
		\EndFunction
	\end{algorithmic}
\end{algorithm}

В алгоритмах \ref{alg5} и \ref{alg6} неоднократно фигурирует функция check(). В данном случае она используется для проверки предложения, не содержащего знаки пунктуации. Её описание будет в следующем параграфе.

Далее возможен один из двух вариантов:
\begin{itemize}
	\item Союз <<и>> связывает только два сочетания.
	\item Союз <<и>> используется для перечисления $3$ и более словосочетаний.
\end{itemize}

В первом случае предложение готово к упрощению: <<кусок>> от left до right заменяем единичным инфинитивом.

Во втором же случае необходимо продолжить анализ предложения, сдвигая левую границу заменяемого участка.

Для начала будем искать участки между двумя запятыми (при их наличии). Особенность данного этапа заключается в том, что между запятыми может оказаться ошибочное словосочетание,~--- потому фрагменты между запятыми нужно также проверять на согласованность.

Также важен порядок рассмотрения случаев: в первую очередь следует искать самые <<короткие>> словосочетания между запятыми (иначе можем <<захватить>> подстроку с запятыми). Этот и последующие этапы описаны в алгоритме \ref{alg7}.

\begin{algorithm}
	\caption{-- Фрагмент алгоритма \ref{alg6}}\label{alg7}
	\begin{algorithmic}[1]
	\While{<<,>>$\in$ l$[1:$ left $]$}\Comment{Пока есть запятые}
	\If{l$[$left $-1]=$ <<,>> \textbf{and} l$[$left $-3]=$ <<,>>} \Comment{Между запятыми одно слово}
	\State res1$\gets$ l$[$left$-2]$.[pos, singular, cow]
	\For{\textbf{from} $i = 0$ \textbf{to} len(res1)}
	\If{res1$[i][0]=$part}
	\State left$\gets$ left$-2$
	\State \textbf{break}
	\EndIf
	\EndFor
	\ElsIf{l$[$left $-1]=$ <<,>> \textbf{and} l$[$left $-4]=$ <<,>>} \Comment{Между запятыми словосочетание из двух слов}
	\State  res1$\gets$ l$[$left$-3]$.[pos, singular, cow]
	\For{\textbf{from} $i = 0$ \textbf{to} len(res1)}
	\If{res1$[i][0]=$part}
	\State r $\gets$ check(l$[$left$-3:$ left$-1 ]$)
	\If{<<N>> $\in$ r}
	\State \Return $[$<<он>>, <<писали>> $]$
	\ElsIf{<<Y>> $\in$ r}
	\State left$\gets$ left$-3$
	\State \textbf{break}
	\EndIf
	\EndIf
	\EndFor
	\ElsIf{l$[$left $-1]=$ <<,>> \textbf{and} l$[$left $-5]=$ <<,>>}
	\State  res1$\gets$ l$[$left$-4]$.[pos, singular, cow]
	\For{\textbf{from} $i = 0$ \textbf{to} len(res1)}
	\If{res1$[i][0]=$part}
	\State r $\gets$ check(l$[$left$-4:$ left$-1 ]$)
	\If{<<N>> $\in$ r}
	\State \Return $[$<<он>>, <<писали>> $]$
	\ElsIf{<<Y>> $\in$ r}
	\State left$\gets$ left$-4$
	\State \textbf{break}
	\EndIf
	\EndIf
	\algstore{bkbreak}
	\end{algorithmic}
	\end{algorithm}

\begin{algorithm}
	\caption{-- Продолжение алгоритма \ref{alg7}}\label{alg8}
	\begin{algorithmic}[1]
		\algrestore{bkbreak}
			\EndFor
			\ElsIf{l$[$left $-1]=$ <<,>> \textbf{and} l$[$left $-6]=$ <<,>>}
		\State  res1$\gets$ l$[$left$-5]$.[pos, singular, cow]
		\For{\textbf{from} $i = 0$ \textbf{to} len(res1)}
			\If{res1$[i][0]=$part}
			\State r $\gets$ check(l$[$left$-5:$ left$-1 ]$)
			\If{<<N>> $\in$ r}
			\State \Return $[$<<он>>, <<писали>> $]$
		\ElsIf{<<Y>> $\in$ r}
			\State left$\gets$ left$-5$
		\State \textbf{break}
		\EndIf
			\EndIf
		\EndFor
		\ElsIf{l$[$left $-1]=$ <<,>> \textbf{and} l$[$left $-7]=$ <<,>>}
		\State  res1$\gets$ l$[$left$-6]$.[pos, singular, cow]
		\For{\textbf{from} $i = 0$ \textbf{to} len(res1)}
		\If{res1$[i][0]=$part}
		\State r $\gets$ check(l$[$left$-6:$ left$-1 ]$)
		\If{<<N>> $\in$ r}
		\State \Return $[$<<он>>, <<писали>> $]$
		\ElsIf{<<Y>> $\in$ r}
		\State left$\gets$ left$-6$
		\State \textbf{break}
		\EndIf
		\EndIf
		\EndFor
		\ElsIf{l$[$left $-1]=$ <<,>> \textbf{and} l$[$left $-8]=$ <<,>>}
		\State  res1$\gets$ l$[$left$-7]$.[pos, singular, cow]
		\For{\textbf{from} $i = 0$ \textbf{to} len(res1)}
		\If{res1$[i][0]=$part}
		\State r $\gets$ check(l$[$left$-7:$ left$-1 ]$)
		\If{<<N>> $\in$ r}
		\State \Return $[$<<он>>, <<писали>> $]$
		\ElsIf{<<Y>> $\in$ r}
		\State left$\gets$ left$-7$
		\State \textbf{break}
		\EndIf
		\EndIf
		\EndFor
		\Else \State Алгоритм \ref{alg9}
		\EndIf
			\EndWhile
		%\algstore{bkbreak}
	\end{algorithmic}
\end{algorithm}
\newpage

Итак, в результате работы фрагментов \ref{alg7} и \ref{alg8} будет сдвинута граница до <<первой>> запятой.

Следующий этап~--- поиск начала перечисления. Соответствующий фрагмент описан алгоритмом \ref{alg9}.

Индикатор end отвечает за нахождение начала перечисления (изначально был инициализирован $(-1)$, а после нахождения начала перечисления будет равен $1$). Как и раньше, проверяем первый найденную подстроку на выполнение правил в ней.
\begin{algorithm}
	\caption{-- Фрагмент алгоритма \ref{alg8}}\label{alg9}
	\begin{algorithmic}[1]
	\If{left$-2\ge 0$\textbf{ and }l$[$left$-1]=$<<,>> \textbf{and} end$=(-1)$}
	\State res1$\gets$ l$[$left$-2]$.[pos, singular, cow]
	\For{\textbf{from} $i = 0$ \textbf{to} len(res1)$ - 1$}
	\If{res1$[i][0]=$<<6>>}
	\State left$\gets$ left$-2$
	\State \textbf{break}
	\EndIf
	\EndFor
	\EndIf
	\If{left$-3\ge 0$\textbf{ and }l$[$left$-1]=$<<,>> \textbf{and} end$=(-1)$}
	\State res1$\gets$ l$[$left$-3]$.[pos, singular, cow]
	\For{\textbf{from} $i = 0$ \textbf{to} len(res1)$ - 1$}
	\If{res1$[i][0]=$<<6>>}
	\State r $\gets$ check(l$[$left$-3:$ left$-1]$)
	\If{<<N>>$\in$ r}
	\State \Return $[$<<он>>, <<писали>> $]$
	\ElsIf{<<Y>> $\in$ r}
	\State left$\gets$ left$-3$
	\State end $\gets 1$
	\State \textbf{break}
	\EndIf
	\EndIf
	\EndFor
	\EndIf
	\If{left$-4\ge 0$\textbf{ and }l$[$left$-1]=$<<,>> \textbf{and} end$=(-1)$}
	\State res1$\gets$ l$[$left$-4]$.[pos, singular, cow]
	\For{\textbf{from} $i = 0$ \textbf{to} len(res1)$ - 1$}
	\If{res1$[i][0]=$<<6>>}
	\State r $\gets$ check(l$[$left$-4:$ left$-1]$)
	\If{<<N>>$\in$ r}
	\State \Return $[$<<он>>, <<писали>> $]$
	\ElsIf{<<Y>> $\in$ r}
	\State left$\gets$ left$-4$
	\State end $\gets 1$
	\State \textbf{break}
\EndIf
	\algstore{bkbreak}
	\end{algorithmic}
\end{algorithm}

\begin{algorithm}
	\caption{-- Продолжение алгоритма \ref{alg9}}\label{alg10}
	\begin{algorithmic}[1]
		\algrestore{bkbreak}
			\EndIf
			\EndFor
			\EndIf
				\If{left$-5\ge 0$\textbf{ and }l$[$left$-1]=$<<,>> \textbf{and} end$=(-1)$}
			\State res1$\gets$ l$[$left$-5]$.[pos, singular, cow]
			\For{\textbf{from} $i = 0$ \textbf{to} len(res1)$ - 1$}
			\If{res1$[i][0]=$<<6>>}
			\State r $\gets$ check(l$[$left$-5:$ left$-1]$)
			\If{<<N>>$\in$ r}
			\State \Return $[$<<он>>, <<писали>> $]$
			\ElsIf{<<Y>> $\in$ r}
			\State left$\gets$ left$-5$
			\State end $\gets 1$
			\State \textbf{break}
			\EndIf
			\EndIf
			\EndFor
			\EndIf
			\If{left$-6\ge 0$\textbf{ and }l$[$left$-1]=$<<,>> \textbf{and} end$=(-1)$}
			\State res1$\gets$ l$[$left$-6]$.[pos, singular, cow]
			\For{\textbf{from} $i = 0$ \textbf{to} len(res1)$ - 1$}
			\If{res1$[i][0]=$<<6>>}
			\State r $\gets$ check(l$[$left$-6:$ left$-1]$)
			\If{<<N>>$\in$ r}
			\State \Return $[$<<он>>, <<писали>> $]$
			\ElsIf{<<Y>> $\in$ r}
			\State left$\gets$ left$-6$
			\State end $\gets 1$
			\State \textbf{break}
			\EndIf
			\EndIf
			\EndFor
			\EndIf
			\If{left$-7\ge 0$\textbf{ and }l$[$left$-1]=$<<,>> \textbf{and} end$=(-1)$}
			\State res1$\gets$ l$[$left$-7]$.[pos, singular, cow]
			\For{\textbf{from} $i = 0$ \textbf{to} len(res1)$ - 1$}
			\If{res1$[i][0]=$<<6>>}
			\State r $\gets$ check(l$[$left$-7:$ left$-1]$)
			\If{<<N>>$\in$ r}
			\State \Return $[$<<он>>, <<писали>> $]$
			\ElsIf{<<Y>> $\in$ r}
			\State left$\gets$ left$-7$
			\State end $\gets 1$
			\State \textbf{break}
			\EndIf
			\EndIf
			\EndFor
			\EndIf
		%\algstore{bkbreak}
	\end{algorithmic}
\end{algorithm}

Итак, после выполнения алгоритма \ref{alg9} будут определены границы заменяемой подстроки, после чего необходимо вставить вместо перечисления инфинитивов одиночный инфинитив. Нами было выбрано слово \textit{<<учить>>} (для данной цели можно было выбрать любой инфинитив, так как мы решаем проблему согласования единственного и множественного числа).

\begin{algorithm}
	\caption{-- Фрагмент алгоритма \ref{alg6}}\label{alg11}
	\begin{algorithmic}[1]
		\State l$\gets$ l$[\,:$ left$] +$ $[$<<учить>>$]+$ l$[$ right$+1:\, $id1$]$
	\end{algorithmic}
\end{algorithm}

Если же при помощи союза <<и>> перечисляются личные глаголы, то упрощение идёт согласно алгоритму \ref{alg12}. В зависимости от длины буквосочетания, возможны различные варианты словосочетаний:
\begin{enumerate}
	\item Словосочетание длины $7$. Например, личн. глаг. + инф. + сущ. + сущ. + сущ.~+ +~сущ.: \textit{<<Хотел организовать проверку знаний требований охраны труда>>}.
	\item Словосочетание длины $6$. Например, личн. глаг. + сущ. + сущ. + сущ. + сущ.~+ +~сущ.: \textit{<<Организовывал проверку знаний требований охраны труда>>}.
	\item Словосочетание длины $5$. Например, личн. глаг. + инф. + сущ. + сущ. + сущ.: \textit{<<Хотел изучить основы теории кодирования>>}.
	\item Словосочетание длины $4$. Например, личн. глаг. + сущ. + сущ. + сущ.: \textit{<<Изучил основы теории кодирования>>}.
	\item Словосочетание длины $3$. Например, личн. глаг. + инф. + сущ.: \textit{<<Желает знать правду>>}.
	\item Словосочетание длины $2$. Например, личн. глаг. + инф.: \textit{<<Желает знать>>}.
	\item Одиночный личный глагол. Например: \textit{<<Желать>>}.
\end{enumerate}
\begin{algorithm}
	\caption{-- Продолжение алгоритма \ref{alg6}}\label{alg12}
	\begin{algorithmic}[1]
		\If{part$=$<<5>>}
		\If{id1$-7\ge 0$ \textbf{and} left$=(-1)$ \textbf{and} <<,>>$\notin$ l$[$id1$ - 7:\,$ id1$]$}
		\EndIf
		\EndIf
		%\algstore{bkbreak}
	\end{algorithmic}
\end{algorithm}
\end{document}