\documentclass[main]{subfiles}





\begin{document}
На стыке лингвистики и computer science в середине XX века возникла компьютерная лингвистика. Это научное направление развивается по мере развития электронно-вычислительных машин \cite{ches}.

Стоит отметить, что попытки исследовать структуру естественного языка математическими методами проводились достаточно давно. Например, известно \cite{mark}, что русский математик Марков Андрей Андреевич, рассматривал распределение доли гласных и согласных в тексте <<Евгений Онегин>>.

Но, как уже было замечено выше, появление и быстрое развитие вычислительной техники повлекло за собой возникновение новой дисциплины~--- компьютерная лингвистика, рождение которой принято связывать с Джорджтаунским экспериментом, когда впервые были продемонстрированы возможности перевода с русского текста на английский и наоборот.

Следующим и очень важным этапом развития дисциплины явилось появление работы <<Синтаксические структуры>> Наома Хомского \cite{synt}, которое послужило основой для создания многочисленных синтетических языков (в том числе языков программирования).

Однако, очень быстро выяснилось, что методы Хомского (прежде всего, рекурсивный грамматический разбор) плохо применимы к естественным языкам. Так, в отличие от искусственных языков, где практически отсутствует проблема многозначности, в любом естественном языке большая часть лексики многозначна: особенно это касается активной части лексики, активного словаря: того, что используется чаще \cite{zz}.

Естественный язык~--- это живое и постоянно развивающиеся явление, тесно связанное с культурными и историческими особенностями своего носителя, которое очень сложно формализовать.

Например, сложно объяснить иностранцу почему у слова \textit{<<река>>} множественное число\textit{ <<реки>>}, а у слова \textit{<<сестра>>} множественное число~--- \textit{<<сёстры>>}.

Итак, постепенно сформировалась новая научная дисциплина~--- Natural Language Processing (NLP).

NLP ставит перед собой весьма амбициозную цель: создание алгоритмов обработки естественного языка, которые понимают и реагируют на текстовые или голосовые данные, отвечая собственным текстом или речью, причём во многом таким же образом, как это делают люди.

Однако, несмотря на очевидные достижения в этой области, остаётся очень много неразрешенных проблем, одной из которых является проблема согласования.

В наиболее общем виде проблема согласования изложена в работе Якова Георгиевича Тестелец <<Введение в общий синтаксис>> \cite{test}.

Согласно его исследованию, в русском языке проблему согласования исходя из синтаксических правил можно разделить на следующие основные части: согласование по времени, роду, падежу и числу.

Например, рассмотрим предложение: <<Я принял решение покусать собаку первым>>.

Можно проиллюстрировать проблему согласования следующими несогласованным предложениями (в скобках приведены примеры низменной синтаксической структуры без потери семантики):
\begin{enumerate}
	\item Согласование по времени: <<Я принял решение покусал собаку первым>>. (<<Я принял решение и покусал собаку первым>>).
	\item Согласование по роду: <<Она принял решение покусать собаку первым>>. (<<Она приняла решение покусать собаку первой>>).
	\item Согласование по числу: <<Мы принял решение покусать собаку первым>>. (<<Мы приняли решение покусать собаку первыми>>).
	\item Согласование по падежу: <<Я принял решение покусать собака первым>>. (<<Я принял решение покусать собаку первым>>).
\end{enumerate}

Стоит отметить, что несмотря на кажущуюся очевидность и простоту проблемы с точки зрения общегуманитарных представлений, алгоритмически она не решена ни для одной из перечисленных выше частей \cite{langt}.

Видимо, в первую очередь это обусловлено тем, что русский язык~--- явление живое, постоянно меняющиеся, где большинство синтаксических правил изобилуют многочисленными исключениями.

И то, что аналоговому прибору под названием человеческий мозг, особенно в период его бурного развития, представляется сравнительно несложной задачей, то <<объяснить>> это машине Тьюринга, пусть и с очень быстрой считывающей головкой и очень длинной лентой, очень даже не просто.

Другими словами, построение алгоритма полностью решающую проблему согласования представляется чрезвычайно сложной задачей.

В данной работе была поставлена \textbf{\textit{цель}}: решить проблему согласования по числу; при этом предполагалось, что предложение состоит из существительных, местоимений, личных глаголов и (или) инфинитивов (с, возможно, единичным перечислением личных глаголов или инфинитивов с зависимыми словами указанных выше частей речи). Это ограничение было введено авторами сознательно, чтобы не утонуть в разборе всех возможных вариантов и решить задачу с максимальной полнотой и точностью.

\end{document}