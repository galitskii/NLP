\documentclass[main]{subfiles}





\begin{document}
\section{Система правил для анализа словосочетаний из двух слов}\label{app:B}

Обозначения:\begin{itemize}
	\item prt~--- часть речи: $1$~--- существительное, $5$~--- личный глагол, $6$~--- инфинитив, $b$~--- местоимение;
	\item sing~--- число: <<N>>~--- множественное, <<Y>>~--- единственное, <<-->>~--- не определено (инфинитивы);
	\item cow~--- падеж: $1$~--- именительный, $2$~--- родительный, $3$~--- дательный, $4$~--- винительный, $5$~--- творительный, $6$~--- предложный, <<-->>~--- не определено (личные глаголы и инфинитивы);
	\item ans~--- ответ: <<Y>>~--- верно, <<N>>~--- неверно.
\end{itemize}
\_r и \_l~--- указатель правого и левого операнда соответственно.
\begin{longtable}[c]{|c|c|c|c|c|c|c|c|p{90px}|}
	%\begin{center}
		\captionsetup{format=hang,labelsep = endash, singlelinecheck=false}
	\caption{Система правил для словосочетаний из двух слов}\\
		%\begin{tabular}
			\hline
			\textbf{№}&\textbf{ prt\_l} &\textbf{sing\_l}  &\textbf{cow\_r} & \textbf{prt\_r }&\textbf{sing\_r }& \textbf{cow\_l} & \textbf{ans}  &\textbf{Пример} \\ \hline
			1 & b & N & 1 & 5 & N & --  & Y  & мы делали \\ \hline
			2 &  1 &Y & 1 & 5 & N & -- & N & собака лаяли\\ \hline
			3 & 1 & Y & 1 & 5 & Y & -- & Y & самолёт летит\\ \hline 
			4 & b & Y & 1 & 5 & Y & -- & Y & я делаю \\ \hline
			5&6&--&--&1&Y&4&Y&делать дело\\ \hline
			6&5&Y&--&6&--&--&Y&хочет есть\\ \hline
			7&6&--&--&b&Y&2&Y&знать его\\ \hline
			8&6&--&--&1&N&5&Y&гордиться детьми\\ \hline
			9&b&Y&1&6&--&--&Y&я есть\\ \hline
			10&b&N&1&6&--&--&Y&вы есть \\ \hline
			11&5&N&--&6&--&--&Y&пришли договориться\\ \hline
			12&b&N&1&5&Y&--&N&мы писал\\ \hline
			13&5&Y&--&b&Y&2&Y&победил меня\\ \hline
			14&5&Y&--&b&N&1&N&вздохнул мы\\ \hline
			15&5&Y&--&1&N&1&N&вздохнул люди\\ \hline
			16&5&Y&--&1&Y&1&Y&бежал человек \\ \hline
			17&5&N&--&1&Y&1&N&бегут собака\\ \hline
			18&5&N&--&1&N&1&Y& бежали собаки\\ \hline
			19&5&N&--&b&Y&1&N&бежали я\\ \hline
			20&5&N&--&b&N&1&Y&бежали мы\\ \hline
			21&6&--&--&b&N&2&Y&укусить нас\\ \hline
	%	\end{tabular}
%	\end{center}
\end{longtable}
\end{document}