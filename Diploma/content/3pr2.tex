\documentclass[main]{subfiles}





\begin{document}
\section{Система правил для анализа словосочетаний из двух слов}\label{app:B}

Обозначения:\begin{itemize}
	\item p~--- часть речи: $1$~--- существительное, $5$~--- личный глагол, $6$~--- инфинитив, $b$~--- местоимение;
	\item s~--- число: <<N>>~--- множественное, <<Y>>~--- единственное, <<-->>~--- не определено (инфинитивы);
	\item c~--- падеж: $1$~--- именительный, $2$~--- родительный, $3$~--- дательный, $4$~--- винительный, $5$~--- творительный, $6$~--- предложный, <<-->>~--- не определено (личные глаголы и инфинитивы);
	\item A~--- ответ: <<Y>>~--- верно, <<N>>~--- неверно.
\end{itemize}
\_r и \_l~--- указатель правого и левого операнда соответственно.
\begin{longtable}[c]{|c|c|c|c|c|c|c|c|p{110px}|}
	%\begin{center}
		\captionsetup{format=hang,labelsep = endash, singlelinecheck=false}
	\caption{Система правил для словосочетаний из двух слов}\label{tab1}\\
		%\begin{tabular}
			\hline
			\textbf{№}&\textbf{ p\_l} &\textbf{s\_l}  &\textbf{c\_r} & \textbf{p\_r }&\textbf{s\_r }& \textbf{c\_l} & \textbf{A}  &\textbf{Пример} \\ \hline
			1 & b & N & 1 & 5 & N & --  & Y  & мы делали \\ \hline
			2 &  1 &Y & 1 & 5 & N & -- & N & собака лаяли\\ \hline
			3 & 1 & Y & 1 & 5 & Y & -- & Y & самолёт летит\\ \hline 
			4 & b & Y & 1 & 5 & Y & -- & Y & я делаю \\ \hline
			5&6&--&--&1&Y&4&Y&делать дело\\ \hline
			6&5&Y&--&6&--&--&Y&хочет есть\\ \hline
			7&6&--&--&b&Y&2&Y&знать его\\ \hline
			8&6&--&--&1&N&5&Y&гордиться детьми\\ \hline
			9&b&Y&1&6&--&--&Y&я есть\\ \hline
			10&b&N&1&6&--&--&Y&вы есть \\ \hline
			11&5&N&--&6&--&--&Y&пришли договориться\\ \hline
			12&b&N&1&5&Y&--&N&мы писал\\ \hline
			13&5&Y&--&b&Y&2&Y&победил меня\\ \hline
			14&5&Y&--&b&N&1&N&вздохнул мы\\ \hline
			15&5&Y&--&1&N&1&N&вздохнул люди\\ \hline
			16&5&Y&--&1&Y&1&Y&бежал человек \\ \hline
			17&5&N&--&1&Y&1&N&бегут собака\\ \hline
			18&5&N&--&1&N&1&Y& бежали собаки\\ \hline
			19&5&N&--&b&Y&1&N&бежали я\\ \hline
			20&5&N&--&b&N&1&Y&бежали мы\\ \hline
			21&6&--&--&b&N&2&Y&укусить нас\\ \hline
			22&6&--&--&5&N&--&N&видеть хотели\\ \hline
			23&6&--&--&5&Y&--&N&быть хотел\\ \hline
			24&1&Y&3&6&--&--&Y&чуду быть\\ \hline
			25&1&N&1&5&N&--&Y&люди делали\\ \hline
			26&1&N&1&5&Y&--&N&люди учил\\ \hline
			27&1&N&3&6&--&--&Y&праздникам быть\\ \hline
			28&b&Y&1&5&N&--&N&я делали\\ \hline
			29&b&Y&2&5&N&--&Y&меня ранили\\ \hline
			30&b&Y&3&5&N&--&Y&мне позвонили \\ \hline
			31&5&N&--&b&Y&2&Y&переиграли меня\\ \hline
			32&5&N&--&b&N&2&Y&позвали нас\\ \hline
			33&5&Y&--&1&N&4&Y&вижу кошек\\ \hline
			34&5&N&--&1&N&2&Y&позвали друзей\\ \hline
			35&1&Y&1&1&Y&3&Y&человек собаке \\ \hline
			36&6&--&--&1&N&2&Y&кормить свиней\\ \hline
			37&6&--&--&6&--&--&Y&хотеть пить\\ \hline
			38&5&Y&--&b&N&2&Y&отчитал их \\ \hline
			39&6&--&--&1&Y&3&Y&дать человеку \\ \hline
			40&1&Y&1&1&Y&4&Y&дурак дурака \\ \hline
			41&6&--&--&b&Y&3&Y&купить себе\\ \hline
			42&6&--&--&b&Y&5&Y&быть собой\\ \hline
			43&5&N&--&1&Y&4&Y&украли сердце\\ \hline
			44&5&Y&--&1&Y&4&Y&кормил собаку \\ \hline
			45&5&Y&--&b&Y&3&Y&помог мне\\ \hline
			46&5&N&--&b&Y&3&Y&помогли мне\\ \hline
			47&6&--&--&1&Y&5&Y&быть учёным \\ \hline
			48&b&Y&1&b&Y&4&Y&он меня\\ \hline
			49&1&Y&4&5&Y&--&Y&человека увидел\\ \hline
			50&b&Y&1&1&Y&4&Y&он руку\\ \hline
			51&1&Y&4&5&N&--&Y&человека ценят\\ \hline
			52&1&N&4&1&Y&1&Y&(губит) людей вода\\ \hline
			53&b&Y&1&1&Y&4&Y&я храм (воздвиг) \\ \hline
			54&1&Y&1&b&Y&3&Y&ребёнок себе (приготовил) \\ \hline
			55&b&Y&3&5&Y&--&Y&себе купил \\ \hline
			56&5&Y&--&b&N&5&Y&играет нами\\ \hline
			57&5&Y&--&b&Y&5&Y&играет тобой\\ \hline
			58&5&N&--&b&Y&5&Y&гордятся тобой\\ \hline
			59&b&Y&1&1&N&4&Y&он руки (моет)\\ \hline
			60&1&N&1&1&Y&4&Y&руки руку \\ \hline
			61&6&--&--&1&N&4&Y&проверять знания\\ \hline
			62&1&N&1&b&Y&4& Y&дети его (знали)\\ \hline
			63&b&Y&1&b&Y&3&Y&ты мне (должен) \\ \hline
			64&5&Y&--&b&Y&1&Y&получился ты\\ \hline
			65&1&Y&2&1&Y&2&Y&певца любви\\ \hline
			66&b&Y&4&5&Y&--&Y&(оно) тебя видит\\ \hline
			67&1&Y&1&1&N&2&Y&игра слов \\ \hline
			68&1&N&1&1&Y&2&Y&порывы ветра\\ \hline
			69&1&Y&4&1&N&2&Y&игру слов\\ \hline
			70&b&Y&2&1&Y&4&Y&его игру\\ \hline
			71&1&Y&5&1&Y&2&Y&сном младенца\\ \hline
			72&b&Y&2&1&N&4&Y&его таблетки\\ \hline
			73&b&N&2&1&Y&4&Y&их игру\\ \hline
			74&5&Y&--&1&Y&3&Y&понравился собаке\\ \hline
			75&5&Y&--&5&Y&--&Y&пойду схожу\\ \hline
			76&1&Y&4&1&Y&2&Y&икру нерки\\ \hline
			77&5&Y&--&b&N&4&Y&звал их\\ \hline
			78&1&N&2&1&N&2&Y&знаний требований\\ \hline
			79&1&N&2&1&Y&2&Y&требований охраны \\ \hline
			80&5&N&--&1&N&5&Y&будьте людьми\\ \hline
	%	\end{tabular}
%	\end{center}
\end{longtable}
Для таблицы \ref{tab2} номера 1, 2 и 3 соответствуют первому, второму и третьему слову соответственно при их чтении слева направо.
\begin{longtable}[c]{|c|c|c|c|c|c|c|c|c|c|c|p{120px}|}
	%\begin{center}
	\captionsetup{format=hang,labelsep = endash, singlelinecheck=false}
\caption{Система правил для словосочетаний из трёх слов}\label{tab2}\\
	%\begin{tabular}
	\hline
	\textbf{№}&\textbf{p1} &\textbf{s1}&\textbf{c1}&\textbf{p2}&\textbf{s2}&\textbf{с2}&\textbf{p3}&\textbf{s3}&\textbf{cow3}&\textbf{A}&\textbf{Пример} \\ \hline
	1 &b  &Y  &1  &1  &Y  &4  &5  &N &--&N&он руку моют \\ \hline
	2&b&Y&1&1&N&4&5&N&--&N&он руки моют\\ \hline
	3&1&N&1&1&Y&4&5&Y&--&N&руки руку моет\\ \hline
	4&b&Y&1&1&N&4&5&Y&--&Y&он руки моет\\ \hline
	5&b&Y&1&b&Y&4&5&Y&--&Y& оно тебя видит\\ \hline
	6&6&--&--&b&Y&2&1&Y&5&Y&жить его жизнью \\ \hline
\end{longtable}
\end{document}