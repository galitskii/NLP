\documentclass[main]{subfiles}


\begin{document}

Проблема согласования единственного и множественного числа не решена; причём, не только в русском языке.

Английский язык <<проще>> тем, что в нём строгий порядок слов: SVO (<<субъект~-- глагол~-- объект>>) \cite{synt}. Однако, даже для английского языка сформулированная проблема не решена. 

Одной из наиболее успешных работ в этой области стала публикация Дамиана Конвея <<An algorithmic approach to English pluralization>> \cite{plur}. В ней автор разрабатывает алгоритмы, преобразующие существительные, прилагательные и глаголы в единственном числе в соответствующие формы множественного числа. Также Конвей приводит алгоритм, позволяющий идентифицировать слова, отличающиеся только числом. Полная реализация данных алгоритмов была сделана автором публикации на языке Perl.

Русский язык, с одной стороны, относится к языкам с фиксированным порядком слов <<SVO>> (как и английский). Однако, с другой стороны, гибкий SV / VS \cite{ox}. За счёт этого задача становится на порядок сложнее. 

Согласование единственного и множественного числа в русском предложении было исследовано в бакалаврской диссертации Дзюбенко Василия Александровича в $2020$ году \cite{dz}. Автором была разработана модель со стеком, совершающую свёртку и сдвиг, аналогичную GLR-анализатору; данная модель позволяла распознавать ошибки согласования в некоторых предложениях.

Тем не менее, В.А. Дзюбенко создал базу данных, в которой содержится информация о подавляющем большинстве слов русского языка. Данная база стала одним из базовых инструментов для решения нами поставленной проблемы.

\end{document}