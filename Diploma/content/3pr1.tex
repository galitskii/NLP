\documentclass[main]{subfiles}





\begin{document}
\section{Структура таблиц базы данных}\label{app:A}
\begin{table}[!h]
	\begin{center}
		\captionsetup{format=hang,labelsep = endash, singlelinecheck=false}
		\caption{words}
		\begin{tabular}{|c|l|l|l|}
			\hline
			\textbf{№} & \multicolumn{1}{c|}{\textbf{Имя столбца}} & \multicolumn{1}{c|}{\textbf{Тип данных}} & \multicolumn{1}{c|}{\textbf{Комментарий}} \\ \hline
			1& word &NVARCHAR2(60) & слово в нижнем регистре\\ \hline
			2&mid& NUMBER(10) & id семейства слов  \\ \hline
			3&fid&NUMBER(6)& id формы слов\\ \hline
			4&cid&NUMBER(6)& id слова данной формы\\ \hline
			5&pos&CHAR(1)& часть речи\\ \hline
			6&singular&CHAR(1)& число \\ \hline
			7&cow &CHAR(1)&падеж \\ \hline
			8&tense &  CHAR(1)& время\\ \hline
			9&kind& CHAR(1)& пол\\ \hline
			10&animal&CHAR(1)&одушевлённость \\ \hline
			11&person&CHAR(1)& лицо \\ \hline
		\end{tabular}
	\end{center}
\end{table}
\begin{table}[!h]
	\begin{center}
		\captionsetup{format=hang,labelsep = endash, singlelinecheck=false}
		\caption{case\_of\_word}
		\begin{tabular}{|c|l|l|l|}
			\hline
			\textbf{№} & \multicolumn{1}{c|}{\textbf{Имя столбца}} & \multicolumn{1}{c|}{\textbf{Тип данных}} & \multicolumn{1}{c|}{\textbf{Комментарий}} \\ \hline
			1&cow &CHAR(1) & id падежа \\ \hline
			2& name& VARCHAR2(30) & название падежа \\ \hline
		\end{tabular}
	\end{center}
\end{table}
\begin{table}[!h]
	\begin{center}
		\captionsetup{format=hang,labelsep = endash, singlelinecheck=false}
		\caption{part\_of\_speech}
		\begin{tabular}{|c|l|l|l|}
			\hline
			\textbf{№} & \multicolumn{1}{c|}{\textbf{Имя столбца}} & \multicolumn{1}{c|}{\textbf{Тип данных}} & \multicolumn{1}{c|}{\textbf{Комментарий}} \\ \hline
			1&pos &CHAR(1) & id части речи \\ \hline
			2& name& VARCHAR2(30) & название части речи \\ \hline
		\end{tabular}
	\end{center}
\end{table}
\begin{longtable}[c]{|c|l|l|p{215px}|}
		\captionsetup{format=hang,labelsep = endash, singlelinecheck=false}
		\caption{simple\_rules}\\ \hline
			\textbf{№} & \textbf{Имя столбца} & \textbf{Тип данных} & \textbf{Комментарий} \\ \hline
			1&r\_id & NUMBER(10)&id правила\\ \hline
			2& prt\_l & VARCHAR2(1)& часть речи левого слова в словосочетании \\ \hline
			3& sing\_l &VARCHAR2(1) & число левого слова в словосочетании \\ \hline
			4& cow\_l &VARCHAR2(1) & падеж левого слова в словосочетании \\ \hline
			5& prt\_r &VARCHAR2(1) & часть речи правого слова в словосочетании \\ \hline
			6& sing\_r &VARCHAR2(1) & число правого слова в словосочетании \\ \hline
			7& cow\_r & VARCHAR2(1)& падеж правого слова в словосочетании \\ \hline
			8&ans&VARCHAR2(1)& ответ\\ \hline
			9&comm &VARCHAR2(1000)& правило\\ \hline
			10 &ex &VARCHAR2(100)&пример  \\ \hline
\end{longtable}

В таблицах \ref{tab4} и \ref{tab5} подразумевается нумерация слов слева направо.
\begin{longtable}[c]{|c|l|l|p{215px}|}
%	\begin{center}
		\captionsetup{format=hang,labelsep = endash, singlelinecheck=false}
		\caption{add3\_r}\label{tab4}\\
%		\begin{tabular}
			\hline
			\textbf{№} & \multicolumn{1}{c|}{\textbf{Имя столбца}} & \multicolumn{1}{c|}{\textbf{Тип данных}} & \multicolumn{1}{c|}{\textbf{Комментарий}} \\ \hline
			1&r3\_id &NUMBER(10) &id правила\\ \hline
			2& prt\_1 &VARCHAR2(1) & часть речи первого слова в словосочетании \\ \hline
			3& sing\_1 &VARCHAR2(1) & число первого слова в словосочетании \\ \hline
			4& cow\_1 &VARCHAR2(1) & падеж первого слова в словосочетании \\ \hline
			5& prt\_2 & VARCHAR2(1)& часть речи второго слова в словосочетании \\ \hline
			6& sing\_2 &VARCHAR2(1) & число второго слова в словосочетании \\ \hline
			7& cow\_2 &VARCHAR2(1) & падеж второго слова в словосочетании \\ \hline
			8& prt\_3 & VARCHAR2(1)& часть речи третьего слова в словосочетании \\ \hline
			9& sing\_3 &VARCHAR2(1) & число третьего слова в словосочетании \\ \hline
			10& cow\_3 &VARCHAR2(1) & падеж третьего слова в словосочетании \\ \hline
			11&ans&VARCHAR2(1)& ответ\\ \hline
			12&comm &VARCHAR2(1000)& правило\\ \hline
			13 &ex &VARCHAR2(100)&пример  \\ \hline
%		\end{tabular}
%	\end{center}
\end{longtable}

\begin{longtable}[c]{|c|l|l|p{215px}|}
%	\begin{center}
		\captionsetup{format=hang,labelsep = endash, singlelinecheck=false}
		\caption{add4\_r}\label{tab5}\\
%		\begin{tabular}{|c|l|l|p{215px}|}
			\hline
			\textbf{№} & \multicolumn{1}{c|}{\textbf{Имя столбца}} & \multicolumn{1}{c|}{\textbf{Тип данных}} & \multicolumn{1}{c|}{\textbf{Комментарий}} \\ \hline
			1&r4\_id & NUMBER(10)&id правила\\ \hline
			2& prt\_1 &VARCHAR2(1) & часть речи первого слова в словосочетании \\ \hline
			3& sing\_1 & VARCHAR2(1)& число первого слова в словосочетании \\ \hline
			4& cow\_1 &VARCHAR2(1) & падеж первого слова в словосочетании \\ \hline
			5& prt\_2 &VARCHAR2(1) & часть речи второго слова в словосочетании \\ \hline
			6& sing\_2 & VARCHAR2(1)& число второго слова в словосочетании \\ \hline
			7& cow\_2 &VARCHAR2(1) & падеж второго слова в словосочетании \\ \hline
			8& prt\_3 &VARCHAR2(1) & часть речи третьего слова в словосочетании \\ \hline
			9& sing\_3 & VARCHAR2(1)& число третьего слова в словосочетании \\ \hline
			10& cow\_3 &VARCHAR2(1) & падеж третьего слова в словосочетании \\ \hline
			11& prt\_4 &VARCHAR2(1) & часть речи третьего слова в словосочетании \\ \hline
			12& sing\_4 &VARCHAR2(1) & число третьего слова в словосочетании \\ \hline
			13& cow\_4 & VARCHAR2(1)& падеж третьего слова в словосочетании \\ \hline
			14&ans&VARCHAR2(1)& ответ\\ \hline
			15&comm &VARCHAR2(1000)& правило\\ \hline
			16 &ex &VARCHAR2(100)&пример  \\ \hline
%		\end{tabular}
%	\end{center}
\end{longtable}

\end{document}