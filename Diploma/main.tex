\documentclass[oneside, final, 12pt]{article}
% пакеты для поддержки языков
\usepackage{cmap} % поиск в PDF
\usepackage[T1, T2A]{fontenc} % кодировка
\usepackage[utf8]{inputenc} % кодировка исходного текста
\usepackage[english,russian]{babel} % локализация и переносы

\usepackage{comment}
%% Гиперссылки
\usepackage[colorlinks,urlcolor=blue]{hyperref}

% основные пакеты
\usepackage{xcolor}
\usepackage{amsmath,amsfonts,amssymb,amsthm,mathtools} % AMS
\usepackage{array,tabularx,tabulary,booktabs} % дополнительная работа с таблицами
\usepackage{graphicx} % для вставки рисунков
\usepackage{subfig} % для нескольких рисунков на одной строке


\usepackage{cite}
\addto\captionsrussian{\def\refname{Список литературы}} %это чтобы было русское название



%\usepackage{biblatex} % библиография
\usepackage{adjustbox}						
\usepackage{tabularx,booktabs}
\usepackage{multicol}
\usepackage{longtable}

\usepackage{newunicodechar}
\newunicodechar{fi}{fi}

% доп. пакеты
\usepackage{mathtext} % русские буквы в формулах
\usepackage{csquotes} % цитирование, в т.ч. русских источников
\usepackage{subfiles} % для сбора из частей
\usepackage{amssymb}
\usepackage{multirow}

% теоремы, определения и пр
\theoremstyle{plain}

%\newtheorem{theorem}{Теорема}[section]


\theoremstyle{plain} % "Примечание"
%\newtheorem{utv}{Утверждение}[section]


% объявляем новую команду для переноса строки внутри ячейки таблицы
\newcommand{\specialcell}[2][c]{%
  \begin{tabular}[#1]{@{}c@{}}#2\end{tabular}}

%% Алгоритмы
\usepackage{algorithm}
\usepackage{algpseudocode}


% НАСТРОЙКА МАКЕТА
% поля страницы
\usepackage{vmargin} 
\setpapersize{A4}
\setmarginsrb{3cm}{1.5cm}{2cm}{2cm}{0pt}{0mm}{0pt}{13mm}


\usepackage{indentfirst} %красная строка у первого параграфа 

% 1.5 межстрочный интервал 
\usepackage{setspace}
\setstretch{1.5}

% 1.25 для абзацного отступа
\setlength{\parindent}{1.25cm} 

% позволяет вставлять разрыв страниц
\allowdisplaybreaks

% установка шрифтов
% шрифты для глав и секций
\usepackage{titlesec}

\titleformat{\section}
{\normalsize\bfseries\centering}
{\thesection.}
{14pt}{}

\titleformat{\subsection}
{\normalsize\bfseries\centering}
{\thesubsection.}
{14pt}{}

\titleformat{\subsubsection}
{\normalsize\bfseries\centering}
{\thesubsubsection.}
{14pt}{}

\floatname{algorithm}{\normalfont Алгоритм}

\makeatletter

\newcommand\appendix@section[1]{\refstepcounter{section}
\orig@section*{{\normalfont ПРИЛОЖЕНИЕ {\@Alph\c@section}\\} #1}%
\addcontentsline{toc}{section}{{\normalfont ПРИЛОЖЕНИЕ \@Alph\c@section. #1}}%
}
\let\orig@section\section
\g@addto@macro\appendix{\let\section\appendix@section}
\makeatother


% путь к библиографии и рисункам
\graphicspath{{images/}}

\usepackage{hyperref}
\hypersetup{
	colorlinks=true,
	citecolor=blue,
	linkcolor=blue,
	urlcolor=blue}

\begin{document}
\setcounter{page}{2}
\section*{Аннотация} 
\subfile{content/5ann}
\pagebreak

\tableofcontents


\pagebreak 

%введение
\section{Введение} 
\subfile{content/0vved}
\pagebreak
\section{Анализ проблемы}
\subfile{content/1lit}
\pagebreak
\section{Описание подхода}
\subfile{content/1algo}

\pagebreak
\section{Заключение}
\subfile{content/4zak}

\pagebreak

\addcontentsline{toc}{section}{\refname}
\bibliography{bib_name}
\begin{thebibliography}{500}
	
%\bibitem{aho} \textbf{Ахо, А. Теория синтаксического анализа, перевода и компиляции}: в 2-х т. / А. Ахо, Дж. Ульман~-- М.: Мир, 1978.~-- Т.1.~-- 613 с.

\bibitem{dz}\textbf{ Дзюбенко, В.А. Согласование единственного и множественного числа в русском предложении}: бакалаврская диссертация: 03.03.01 / Дзюбенко Василий Александрович.~-- Долгопрудный, 2020.~-- 20 с.

\bibitem{forms}\textbf{ Журавлёв, Ю.И. Дискретный анализ. Формальные системы и алгоритмы}: Учебное пособие / Ю.И. Журавлёв, Ю.А. Флёров, Н.М. Вялый~-- М.: ООО Контакт Плюс, 2010.~--~ 336 с.

%\bibitem{tryap} \textbf{Теория и реализация языков программирования}: Учебное пособие / В.А. Серебряков [и др.].~-- М.: МЗ Пресс, 2006.~-- 352 с.

\bibitem{ches} \textbf{Чесебиев, И. А. Компьютерное распознавание и порождение речи}: монография.~-- Москва: Спорт и Культура-2000, 2008.~-- 125 с.

\bibitem{synt} \textbf{Comrie, B. Language universals and linguistic typology: Syntax and morphology}.~-- University of Chicago press, 1989.

\bibitem{plur} \textbf{Conway, D. An algorithmic approach to English pluralization} // Proceedings of the Second Annual Perl Conference.~-- 1998.

\bibitem{ox} \textbf{The world atlas of language structures} / M. Haspelmath [and others].~-- Oxford Univ. Press, 2005.

\bibitem{pym} Морфологический анализатор pymorphy2 [Электронный ресурс]~--- \href{https://pymorphy2.readthedocs.io/en/latest/}{https://pymorphy2.readthedocs.io/en/latest/}

\bibitem{langt} LanguageTool~--- Проверка грамматики и стилистики [Электронный ресурс]~--- \href{https://languagetool.org/ru}{https://languagetool.org/ru}


\end{thebibliography}
\pagebreak

\appendix

%Приложения
\subfile{content/3pr1}
\pagebreak
\subfile{content/3pr2}

\end{document}